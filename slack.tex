\title{Using Slack for the \texttt{Systems Security} course}
\date{}

\documentclass[10pt]{article}
\usepackage{hyperref}

\begin{document}
\maketitle

There is a \texttt{Slack} organisation setup for the course.
You should use it to communicate with the Lecturer and the Teaching Assistants.
You should be able to register using your University of Bristol e-mail at the following address: \url{https://bris-sys-sec.slack.com/}.

Please, prefer slack over e-mails. However, do not discuss personal matters or send coursework (even draft) through \texttt{Slack}.
You should use e-mail or discuss such matters in person with the Lecturer/TAs.

\vspace{1cm}
Please also keep in mind the following rules:
\begin{enumerate}
	\itemsep0em
	\item Use your real name (no \texttt{DarkLordOfDoom666}).
	\item Use a picture of yourself as avatar (as above).
	\item Keep it civil and cordial.
	\item No private messaging to Lecturer/TAs. If you are not comfortable discussing matters in public, prefer a face to face chat.
	Private messages will be ignored.
	\item You can answer your classmate questions, but do not do their work for them (see labs instructions). This includes posting code, scripts, report material etc.
	Be aware that plagiarism will not be tolerated.
	\item There is a channel per lab; send your questions to the most appropriate channel. 
	There is a specific channel for questions regarding the lectures.
	\item There will be office hours. 
	Office hours should not be where you ask basic questions about the coursework (e.g., my script is not working).
	However, feel free to come discuss about the reflection part of the coursework.
\end{enumerate}


\end{document}
